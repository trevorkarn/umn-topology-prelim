\documentclass{article}

\usepackage{amsthm}
	\newtheorem*{definition}{Definition}
	\newtheorem*{theorem}{Theorem}
	\newtheorem*{lemma}{Lemma}
\usepackage{amsmath}
\usepackage{amsfonts}
\usepackage{amssymb}
\usepackage[margin=1in]{geometry}
\usepackage{hyperref}
\usepackage{tikz}
	\usetikzlibrary{cd}
	\usetikzlibrary{patterns}
	
\DeclareMathOperator{\Tor}{Tor}
\DeclareMathOperator{\im}{im}
\renewcommand{\theequation}{\roman{equation}}

\title{\href{https://math.umn.edu/sites/math.umn.edu/files/exams/mantops16.pdf}{Spring 2016 Manifolds and Topology Preliminary Exam}}
\author{University of Minnesota}
\date{}
\begin{document}
\maketitle

\section*{Part A}
\begin{enumerate}
	\item 
%		\begin{enumerate}
%		
%		\item Suppose $X$ is a space with a continuous product $\mu : X \times X \rightarrow X$
%			together with an element $e \in X$ such that $\mu(x,e) = \mu(e,x) = x$ for all $x \in X$. 
%			For loops $\alpha, \beta : [0,1] \rightarrow X$ starting and ending at $e$, define
%			\[ \alpha \circ \beta (t) = \mu( \alpha(t), \beta(t)) \]
%			Suppose that $*$ is the ordinary concatenation product on loops. Show there is an 
%			identity 
%			\[ (\alpha * \beta) \circ ( \gamma * \delta) = (\alpha \circ \gamma) * (\beta \circ \delta). \]
%		
%		\item Suppose $\circ$ is the product from the previous problem. Show that it determines a well-defined
%		product $\circ : \pi_1(X,e) \times \pi_1(X,e) \rightarrow \pi_1(X,i)$ with unit $c_e$ the trivial loop at $e$.
%		
%		\item Use the previous results to conclude that $\circ = *$ and both products are commutative
%		
%		\item Describe the fundamental group of the Klein bottle using generators and relations.
%		
%		\end{enumerate}
%	
	\item
	
	\item 
		\begin{enumerate}
		
		\item For $n >0$ define the degree of a continuous map $S^n \rightarrow S^n$.
		\begin{definition}
		If $n>0$ and $f:S^n \rightarrow S^n$ is continuous map, then the induced map on homology
		$f_*: H_n(S^n) \rightarrow H_n(S^n)$ is a homomorphism
		$f_*: \mathbb{Z} \rightarrow \mathbb{Z}$.
		The only homomorphism from $\mathbb{Z} \rightarrow \mathbb{Z}$ is multiplication by a constant
		integer, so $f_*(x) = dx$ for $x \in H_n(S^n)$ and $d \in \mathbb{Z}$.
		Then $d$ is the degree of $f$.
		\end{definition}
		
		\item If $M$ is an $n$-dimensional manifold with $n > 0$ and $p \in M$, show that
		there is an isomorphism of homology groups
		\[ H_k(M, M\backslash \{ p\}) \cong \begin{cases} \mathbb{Z} & \text{ if $n = k$ }\\ 0 & \text{otherwise} \end{cases} \]
		
		\begin{proof}
		
		Since $M$ is a manifold, we may take an open neighborhood $U_p \cong \mathbb{R}^n$.
		Define $C = M- U_p$. Then $M-C = U_p$ and $(M-p) - C = U_p - p$.
		 The excision theorem tells us that
		\[H_*( M, M - p ) \cong H_*(M - C, (M-p)- C) = H_*(U_p, U_p-p)\]
		The long exact sequence on relative homology tells us
		\[\begin{tikzcd}
		\cdots \arrow[r] & \tilde{H}_k(U_p - p) \arrow[r] & \tilde{H}_k(U_p) \arrow[r] & \tilde{H}_k(U_p, U_p-p) \arrow[r] & \tilde{H}_{k-1}(U_p-p) \arrow[r] & \cdots.
		\end{tikzcd}\]
		We know that $U_p - p$ is homotopic to the $n-1$ sphere and since homology is a homotopy invariant, we get that
		\[\tilde{H}_k(U_p - p ) = \tilde{H}_k(S^{n-1}) = \begin{cases} \mathbb{Z} & {k=n-1} \\ 0 & \text{otherwise} \end{cases} \]
		and so when $k \neq n-1$, we have that 
		$\tilde{H}_k(U_p) \cong \tilde{H}_k(U_p, U_p-p) $. Moreover $\tilde{H}_k(U_p) = 0$ for all $k$, and reduced homology agrees with
		singular homology for $k>0$, so we have so far that
		\[H_k(U_p, U_p-p) = \begin{cases} 0 & k \neq 0, n-1 \\ ?? & \text{otherwise} \end{cases}\]
		Now, focusing near $k=n-1$ we have the exactness of the sequence
		\[\begin{tikzcd}
		\cdots \arrow[r] & H_n(U_p) \arrow[r] & H_n(U_p, U_p-p) \arrow[r]  & H_{n-1}(U_p - p) \arrow[r] & H_{n-1}(U_p) \arrow[r] & \cdots.
		\end{tikzcd}\]
		Since $H_n (U_p) = H_{n-1}(U_p) = 0$ the above sequence induces the isomorphism
		$H_n(U_p, U_p-p) \cong H_{n-1}(U_p - p) \cong \mathbb{Z}$.
		Finally, near $0$ we have
		\[ \begin{tikzcd}
		\cdots \arrow[r] & H_0 (U_p) \arrow[r] & H_0 (U_p, U_p-p) \arrow[r] & 0
		\end{tikzcd} \]
		which is exactly
		\[ \begin{tikzcd}
		\cdots \arrow[r] & 0 \arrow[r] & H_0 (U_p, U_p-p) \arrow[r] & 0
		\end{tikzcd} \]
		and so $H_0 (U_p, U_p-p) = 0$
		\end{proof}

		
		\item Show that the subspace
		\[ \{(x,y,z) | \text{either $(x=0)$ or $(y=z=0)$ }\} \subset \mathbb{R}^3 \]
		is not a manifold.
		
		\begin{proof}
		Suppose the subspace \emph{were} a manifold. Then in the neighborhood of $(0,0,0)$, we can find $U_0 \cong \mathbb{R}^n$ for some $n$ (perhaps the manifold is impure).
		Then $U_0 - (0,0,0)$ is homotopic to an $n-1$ sphere. When $n \geq 1$, this means that $U_0 - (0,0,0)$ has one connected component.
		On the other hand when $n = 0$, $U_0 - (0,0,0)$ has two connected components.
		
		The subspace is the union of the $x$ axis with the $y-z$-plane.
		But we see that any neighborhood of $(0,0,0)$ in the subspace given can be broken into three disconnected components: 
		the intersection with the positive $x$-axis, the negative $x$-axis, and the punctured $y-z$-plane missing the origin.
		Thus every neighborhood of the origin is \emph{not} homeomorphic to Euclidean space, and so the subspace is not a manifold.
		\end{proof}
		
		\item For $n \in \mathbb{Z}$ give an example of a continuous map $S^1 \rightarrow S^1$ of 
		degree $n$.
	
		\begin{proof}
		Let $f:S^1 \rightarrow S^1$ be given by $z \mapsto z^n$, where we implicitly think of $S^1 = \{ z \in \mathbb{C} : |z|=1 \}$.
		We claim that the degree of $f$ is $n$. 
		Let $z = e^{i \theta}$ where $\theta \in \mathbb{R}$.
		The fiber $f^{-1}(z) = \{ z \in S^1 : z=e^{i(\theta + 2\pi k)/n}, k \in \mathbb{Z}\}$ consists of $n$ points
		$Z = \{ e^{i \theta/n} , e^{i \theta/n + i 2\pi 1/n},..., e^{i \theta/n + i 2\pi (n-1)/n}  \}$.
		$f$ is orientation preserving, and so $\deg f |_{z} = 1$ for all $z \in Z$.
		Then $\deg f = \sum_{z \in Z} \deg f |_{z} = n$. 
		\end{proof}		
		\end{enumerate}
	
\end{enumerate}

%\section*{Part B}
%\begin{enumerate}
%	\item 
%	
%	\item
%	
%	\item
%	
%\end{enumerate}
\end{document}
