\documentclass{article}

\usepackage{amsthm}
	 \newtheorem*{theorem}{Theorem}
	 \newtheorem{proposition}{Proposition}
	 \newtheorem*{lemma}{Lemma}
\usepackage[margin=1in]{geometry}
\usepackage{tikz}
	\usetikzlibrary{cd}

\title{Useful theorems for Manifolds and Topology Preliminary Exams}
\author{University of Minnesota}

\begin{document}

\maketitle

\section{Fundamental Group}

\begin{theorem}[Siefert-van Kampen]

Let $U,V$ be open, path connected topological spaces such that $U \cap V$ is nonempty and path connected. The inclusion maps of $U \hookrightarrow U \cup V$ and $V \hookrightarrow U \cup V$ induce group homomorphisms $j_U:\pi_1(U) \rightarrow \pi_1(U \cup V)$ and $j_V: \pi_1(V) \rightarrow \pi_1(U \cup V)$. Then $U \cup V$ is path connected, and $j_U, j_V$ form a commutative pushout diagram:
			
			\begin{center}\begin{tikzcd}
				& \pi_1(U) \arrow[rd,dashed] \arrow[rrd, bend left=10,"j_U"]& \\
				\pi_1(U \cap V) \arrow[ru,"i_U"] \arrow[rd,"i_V"] & & \pi_1(U) *_{\pi_1(U \cap V)}\pi_1(V) \arrow[r,dashed,"k"] & \pi_1(U \cup V) \\
				& \pi_1(V) \arrow[ru,dashed] \arrow[rru, bend right=10,"j_V"]&
			\end{tikzcd}\end{center}

Since this is a pushout diagram, then $k$ is an isomorphism.

\end{theorem}

\section{Covering Spaces}

\begin{theorem}
Let $X$ be path connected, locally path connected, and semilocally simply connected. Then there is a bijection between the set of basepoint-preserving isomorphism classes of path-connected covering spaces $p:(\tilde{X},\tilde{x}_0) \rightarrow (X,x_0)$ and the set of subgroups of $\pi_1(X,x_0)$ obtained by associating the subgroup $p_*(\pi_1(\tilde{X}, \tilde{x}_0))$ to the covering space $(\tilde{X}, \tilde{x}_0)$. If basepoints are ignored, this gives a bijection between isomorphism classes of path-connected covering spaces $p: \tilde{X} \rightarrow X$ and conjugacy classes of subgroups of $\pi_1(X,x_0)$.
\end{theorem}

\begin{lemma}
	If $G$ is an abelian group, then the conjugacy classes of $G$ are all singletons, so if $G$ is finite, then $|G|$ is the number of conjugacy classes.
\end{lemma}

\end{document}